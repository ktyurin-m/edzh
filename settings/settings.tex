\documentclass[a4paper,14pt]{extarticle}
%%% Работа с русским языком
\usepackage{cmap}					% поиск в PDF
\usepackage{mathtext} 				% русские буквы в фомулах
\usepackage[T2A]{fontenc}			% кодировка
\usepackage[utf8]{inputenc}			% кодировка исходного текста
\usepackage[english,russian]{babel}	% локализация и переносы
\usepackage[unicode, pdftex]{hyperref}
% \usepackage[unicode=true]{hyperref}
%%% Дополнительная работа с математикой
\usepackage{amsmath,amsfonts,amssymb,amsthm,mathtools} % AMS
\usepackage{icomma} % "Умная" запятая: $0,2$ --- число, $0, 2$ --- перечисление
\usepackage{graphicx}%Вставка картинок правильная

\usepackage{float}%"Плавающие" картинки

\usepackage{wrapfig}%Обтекание фигур (таблиц, картинок и прочего)
%% Номера формул
\mathtoolsset{showonlyrefs=true} % Показывать номера только у тех формул, на которые есть \eqref{} в тексте.
\usepackage{subcaption} 
\usepackage{tabularx}
%% Шрифты
\usepackage{euscript}	 % Шрифт Евклид
\usepackage{mathrsfs} % Красивый матшрифт
\captionsetup[figure]{labelsep = period}
% \usepackage{indentfirst}

\usepackage{listings}
\lstset{
    extendedchars=\true,
    commentstyle=\itshape,
    stringstyle=\bfseries,
    keepspaces=true,
  escapechar=\%,
  texcl=true
  showstringspaces=false,
  numbers=left,
  breaklines,
  columns=fullflexible,
  flexiblecolumns,
  numberstyle={\footnotesize},
  frame=tb
}
% \renewcommand{\theequation}{\arabic{subsection}.\arabic{equation}}
%Задание параметров страницы
% в преамбуле
% \usepackage[a4paper, mag=1000, left=2.5cm, right=1cm, top=2cm, bottom=2cm, headsep=0.7cm, footskip=1cm]{geometry}
\usepackage[left=3cm, right=1cm, top=2cm, bottom=2cm]{geometry}
\usepackage[onehalfspacing]{setspace}
\usepackage{caption}    
\usepackage{subcaption}       
\renewcommand{\thesubfigure}{\asbuk{subfigure}}           % Буквенные номера подрисунков
\captionsetup[subfigure]{font={normalsize},               % Шрифт подписи названий подрисунков (не отличается от основного)
    labelformat=brace,                                    % Формат обозначения подрисунка
    justification=centering,                              % Выключка подписей (форматирование), один из вариантов            
}
\numberwithin{equation}{section}


